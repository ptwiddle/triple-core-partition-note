\documentclass{amsart}

\usepackage{mathpazo}
\usepackage{microtype}

\linespread{1.2}

\theoremstyle{definition}
\newtheorem{theorem}{Theorem}
\newtheorem{conjecture}{Conjecture}
\newtheorem{lemma}{Lemma}
\newtheorem{definition}{Definition}
\newtheorem{corollary}{Corollary}

\newcommand{\Mot}{\mathbf{Mot}}
\newcommand{\SC}{\mathbf{SC}}
\newcommand{\MP}{MP}
\newcommand{\Cat}{\mathbf{Cat}}

\title{Simultaneous cores and Rational Motzkin numbers}
\author{Paul Johnson}

\begin{document}
\maketitle
Let $\lambda$ be a partition, and $\square\in\lambda$ a cell of its diagram.
The hook length $h(\square)$ is , a partition $\lambda$ is an $a$-core if it has no hook lengths of size $a$.


In \cite{johnson}, we used lattice point geometry as a tool for studying partitions that are simultaneously $a$-core, and $b$-core.  There has recently been interest in partitions that are simultaneous core for more than two numbers.  This note is an addendum to \cite{johnson}, showing the techniques developed there are useful in the general case as well.  

As a side benefit, we point out that the lattice point techniques are not just useful for studying cores without hooklengths of another size, but counting the hooklengths of these other sizes.


Our starting point is the following conjecture of Amdeberhan:
\begin{conjecture}[Amdeberhan] \label{conj:Amdeberhan}
Suppose $s,d$ are relatively prime.
Then the number of $(s,s+d,s+2d)$-core partitions is given by:
$$\sum_k \frac{1}{s+d}\binom{s+d}{k,d+k,s-2k}$$
\end{conjecture}
Our first result proves a refined version of this conjecture, which provides an interpretation of each term appearing in the sum:

\begin{theorem} \label{thm:cores}
Let $s,d$ be relatively prime.  Then the number of $(s,s+d,s+2d)$ cores with exactly $k$ hooklengths of size $d$ is
$$\sum_k \frac{1}{s+d}\binom{s+d}{k,d+k,s-2k}$$
\end{theorem}

As a byproduct of this result, we noticed that the lattice point methods of \cite{johnson} allow us not only to consider $a$-cores \emph{without} any hooklengths of a given size, but more generally to \emph{count} the number of hook lengths of a given size in any $a$-core.  

We do not develop and explore all the consequences of this observation, but summarize it as follows:

\begin{theorem}
Let $n_k(\lamda)$ denote the number of hooklengths of size $k$ of a partition.
Within the lattice of $a$-cores, $n_k(\lambda)$ is an explicit piecewise linear function.
\end{theorem}

We also observe that this lets us rephrase the \emph{skew length} function introuduced by \cite{ } and studied in \cite{ } is expressable in terms of the $n_k$:
$$s\ell_{a,b}=\sum_{k=1}^{a-1} (n_k-n_{b+k})$$

\subsection{} Our second result provides another combinatorial object counted by these numbers.

The starting point for studying simultaneous core partitions is Anderson's result that the number of $(a,b)$ cores is the \emph{rational Catalan number}
$$\Cat_{a,b}=\frac{1}{a+b}\binom{a+b}{a}$$
Anderson gave a bijection between simultaneous core partitions and an object already known to be counted by rational Catalan numbers, \emph{rational Dyck paths}:
\begin{definition}
Let $a,b$ be relatively prime.  A \emph{rational Dyck path} is a path from $(0,a)$ to $(b,0)$ that stays beneath the line connecting those two points consiting only of steps of size $(0,-1)$ and $(1,0)$.
\end{definition}

Amdeberhan and Leven \cite{AL} introduce the notion of a generalized Dyck path, and showed that the number of $(s,s+1,\dots, s+t)$-cores are counted by these.  

  In particular, their 2,  are already known as Motzkin paths.  We generalize the notion of Motzkin path to a \emph{rational Motzkin path}, and show that these have the same count as $(s,s+d,s+2d)$-cores.

\begin{definition} Let $a+b$ be even, and $(b-a)/2$ be relatively prime to $a$.
A  \emph{rational Motzkin path} from $(a,0)$ to $(0,b)$ is a path staying below the line that connects them, of steps of the form $S=(0,-2), D=(1,-1)$ or $E=(0,2)$ (for \emph{South}, \emph{Diagonal}, and \emph{East}).  

We also call these $(a,b)$-Motzkin paths, and we call the number of such paths the \emph{rational Motzkin number} $\Mot_{a,b}$.
\end{definition}

\begin{theorem} \label{thm:paths}
Let $a+b$ be even, and $(b-a)/2$ relatively prime to $a$.  The number of $(a,b)$-Motzkin paths with exactly $k$ down steps is
$$\frac{1}{s+d}\binom{s+d}{k,d+k,s-2k}$$
\end{theorem}

\begin{corollary}
The number of $(s,s+d,s+2d)$-cores is equal to the number of $(s,s+2d)$-Motzkin paths.
\end{corollary}



\section{Proof of Theorem \ref{thm:cores}}

The methods of \cite{johnson} are easily extended to give the following:
\begin{lemma} \label{lem-lattice}
The number of $(s,s+d,s+2d)$-cores is $1/s$ times the number of $s$-tuples of nonnegative integers $z_i$ that sum to $s+d$, and no two cyclicly consecutive $z_i$ are equal to 0.

Furthermore, the number of cells with hooklength $d$ in such a partition is equal to the number of $z_i$ are that are equal to 0.
\end{lemma}
\begin{proof}
The output of \cite{johnson} gives that the number of $(s,s+d,s+2d)$-cores is $1/s$ times the number of $s$-tuples $(z_1,\dots,z_s)$ of non-negative integers, such that $\sum_{i=1}^s z_i=s+d$ and $z_i+z_{i+1}\geq 1$ for all $i$, where the indicies are taken mod $s$ -- i.e., we also require $z_s+z_1\geq 1$.

Without the last condition on $z_i+z_{i+1}$, the solutions to this are lattice points in the $s+d$ scaled standard simplex, and $1/s$ of them count the $(s,s+d)$ simultaneous cores, as explained at length in \cite{Johnson}.   This last condition is easily seen to be the one that implies that it is also an $s+2d$ core -- first, write this condition in the $x$ coordinates and then translate into the $z$ coordinates.

For each $i$ with $z_i=0$, we see that there is a pair of runners on the abacus presentation with the distance between the highest filled energy level on one runner and the lowest vacant energy level on the other is exactly $d$, which corresponds to a cell with hooklength $d$.    
\end{proof}



Theorem \ref{thm-cores} follows Lemma \ref{lem-lattice} and
\begin{lemma}
The number of $s$ tuples of non-negative integers summing to $s+d$ with exactly $k$ of the integers equal to zero, and none of these zeros being cyclically adjacent is 
$$\frac{s}{s-2k}\binom{s-k-1}{k}\binom{d+s-1}{s-k-1}$$
\end{lemma}

\begin{proof}
It is standard that the second binomial coefficient is the number of ways that $s-k$ positive integers can sum to $d+s$.  We prove that the other factors count the number of ways to choose $k$ of the $z_i$ to be 0, subject to the restriction that none of the zeroes are cyclically adjacent.

There are $s$ ways to choose one of the entries to be zero.  We remove this entry, ``cutting'' the cycle of entries and arranging the remaining $s-1$ entries linearly, we need to choose $k-1$ of them to be zero, with the zero entries not allowed to be adjacent or the first or last entry.  Thus, there will be $k$ nonempty strings of nonzero entries.  Subtracting 1 from the length of each,  there are $\binom{s-k-1}{k-1}$ ways to choose the $0$s.  Our initial choice of an entry to be 0 could be any one of the $k$ 0 entries, and so we are counting each arrangement $k$ times, and there are
$$\frac{s}{k}\binom{s-k-1}{k-1}$$ ways to choose the entries that are zero.
\end{proof}

\begin{definition} 
The \emph{Motzkin triangle} $M(n,k)$ is defined to be the number of paths from $(0,0)$ to $(n, n-k)$, consisting of $n$ steps in the directions $\{(1,1), (1,0), (1,-1)\}$ that stay above $x$-axis.

Alternatively, we have $M(n,0)=1, M(n+1,k)=M(n,k)+M(n,k-1)+M(n,k-2)$.
\end{definition}

\begin{theorem}[Ambederhan, , Zeilberg]
$$\sum_k \binom{s+d-1}{2k+d-1}\binom{2k+d}{k}\frac{1}{2k+d}=\frac{1}{d} M(s+d-1,s)$$

\end{theorem}

\section{Proof of Theorem \ref{thm:paths}}
The proof follows the usual one that rational Catalan numbers count rational Dyck paths, with the number theory at the end being slightly more complicated.


A $(s,s+2d)$-Motzkin path travels a distance of $2s+2d$ in the taxi-cab metric, out of steps that are each of length $2$, and thus consists of $s+d$ total steps.  If $k$ of them are down, then $s-2k$ of them must be diagonal, and hence the remaining $d+k$ of them must be to the right.  

Thus, the multinomial coefficient counts the number of Motzkin paths if we ignore the condition that we remain below the diagonal.  There is an action $\Z/(s+d)/Z$ on such paths by cyclicly permuting the steps.  We show that each orbit under this action contains exactly one path that stays below the diagonal.

Label each lattice point $(i,j)$ with a \emph{height} $H(i,j)=si+(s+2d)j$.  The lattice points $(0,s)$ and $(s+2d,0)$ each have height $s^2+2ds$, and so the lattice points below the line connecting them are exactly those with height less than this.

We claim any path between these two points has a unique point of highest height, unless it's a rational Dyck path, in which case the start and end point each have the same height.  This proves the result -- simply cyclic rotate the highest point on the path to be the starting (and ending) point.

If a path doesn't have a unique highest point, then the subpath connecting two different points of maximum height changes the height by $0$.  Suppose this subpath consists of $\sigma$ steps $S$, $\delta$ steps $D$, and $\epsilon$ steps $E$.We show that $\sigma+\delta+\epsilon\geq s+d$, which shows that in this case the maximum height occurs at the start and finish, as desired.  

  Since each $S$ decreases the height by $2s+4d$, each $D$ decreases the height by $2d$, and each $E$ increases the height by $2s$, dividing by two gives $$(s+2d)\sigma+d\delta=s\epsilon$$
Since $s$ and $d$ are relatively prime, for the left hand side to be divisible by $s$, $2\sigma+\delta$ must be disivible by $s$.  Thus, the minimum nonzero solution has $2\sigma+\delta=s$, making the right hand side equal to $s(\sigma+d)$, and so $\epsilon=\sigma+d$.

Thus, $\sigma+\delta+\epsilon=2\sigma+\delta+d=s+d$, as desired.



\end{document}
